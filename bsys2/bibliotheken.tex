\section{Bibliotheken}

%\subsection{Statische Bibliotheken} 
%Archiv von Obj-Dateien, wie mehrere Obj-Dateien behandelt lib$<$name$>$.a\\
%\textcolor{green}{+}einfache Verwendung/Implementation
%\textcolor{red}{-}Neuerstellung der Programme bei Änderungen der Lib nötig
%\textcolor{red}{-}fixe Funktionalität

%\subsection{Dynamische Bibliotheken}
%Executable nur noch Referenz auf Bibliothek, zur Lade-/Laufzeit gelinkt
%\textcolor{green}{+}Programm muss nur Bibliothek laden, die es braucht

%\subsection{API}
%\begin{minted}{c}
%//Öffnet dynamische Bibliothek, Handle zurück
%void* dlopen(char* filename, int mode)
%//dlsym Adresse des Symbols als void*
%typedef int (*funct_t)(int);
%funct_t f = dlsym(handle, "my_function");
%int *i = dlsym(handle, "my_int");
%(*f)(*i);
%int dlclose(void* handle);
%char dlerror();
%\end{minted}

%\subsection{Shared Objects}
%Referenz in Executable nötig, OS sucht bei Programmstart automatisch richtige Bibliotheken\\
%Versionen und Unterversionen können gleichzeitig verwendet werden/existieren

\subsubsection{Benennungsschema}
\begin{tabular}{|l|l|l|}
\hline
   Linker-Name:  & lib + Bibliotheksname + .so & libmylib.so\\
   \hline
   Shared Object-Name: & Linker-Name + . + Vers.nr. & libmylib.so.2\\
   \hline
   Real-Name: & SO-Name + . + Untervers.nr. & libmylib.so.2.1\\
   \hline
\end{tabular}
Shared Object-Name für Loader\\
\prgc{/usr/lib} + Linker-Name, Softlink auf $\rightarrow$ \prgc{/usr/lib} + SO-Name, Softlink auf $\rightarrow$ absoluter Speicherort \prgc{/usr/lib} + Real-Name\\
Versionnr. erhöhen, wenn Schnittstelle ändert\\
Unterversionsnr. erhöhen, wenn Schnittstelle bleibt (Bugfixes)

\subsection{Implementierung}
Dynamische Bibliotheken müssen verschiebbar sein\\
Code zwischen Programmen soll geteilt werden: nur einmal im Hauptspeicher $\rightarrow$ Shared Memory\\
Anwendung: Virtuelle Pages der Prozesse werden auf denselben Frame im RAM gemappt $\rightarrow$ Adressen müssen relativ sein! (position-independent)

%\subsubsection{Position-Independent}
%Relative Adressen zum Instruction Pointer\\
%ist somit egal, wo im Speicher der Code liegt\\
%X86\_64: relative Calls \& Move-Instr.
%x86\_32: nur relative Calls

%\subsubsection{Relative Moves via Relative Calls}
%Funktion f will relativen Move ausführen:
%\begin{enumerate}
%    \item Aufruf der Helperfunktion h durch f
%    \item h kopiert Rücksprungsadresse vom Stack in Register \& springt zurück
%\end{enumerate}
%$\rightarrow$ f hat Rücksprungsadresse = IP in Register, kann relativ dazu arbeiten

\subsubsection{Global Offset Table (GOT)}
eine pro dynamischer Bibliothek/Executable\\
pro Symbol, das von anderer dynamischer Bibliothek benötigt wird, ein Eintrag\\
Im Code werden relative Adressen in die GOT verwendet.\\
Loader füllt zur Laufzeit "echte" Adresse in GOT ein.

\subsubsection{Procedure Linkage Table (PLT)}
implementiert Lazy Binding(Funktionen werden erst gebunden, wenn benötigt)\\
pro Funktion ein Eintrag\\
PLT-Eintrag enthält Sprungbefehl auf Stelle in GOT\\
GOT-Eintrag zunächst Proxy-Funktion\\
Proxy-Funktion sucht Link zu richtiger Funktion, überschreibt dann eigener GOT-Eintrag\\
Vorteil: Erspart bedingten Sprung
