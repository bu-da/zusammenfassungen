\section{OS API}
\subsection{Aufgaben OS}
\begin{itemize}
    \item Abstraktion/Portabilität von Hardware, Protokolle, Software-Services
    \item Resourcenmanagement/Isolation der Anwendungen voneinander (Rechenzeit, Hauptspeicherverwaltung, Sek. Speicher, Netzwerkbandbreite)
    \item Benutzerverwaltung/Sicherheit
\end{itemize}
\subsection{Prozessor Privilege Level}
mind. 2 Privilege Levels auf Prozessor: Kernel Mode, User Mode\\
Kernel bestimmt in welchem Modus ein Programm läuft (Entscheid somit softwareseitig)
%\textbf{Micro-Kernel:} selbst Gerätetreiber in User Mode \textcolor{green}{+} Stabilität, Analysierbarkeit \textcolor{red}{-} Performance (häufige Mode-Wechsel)
%\textbf{Monolithisch (Linux etc.):} sehr viel Funkt. in Kernel, welche nicht nötig wären \textcolor{green}{+} Performance (weniger Wechsel) \textcolor{red}{-}Programmierfehler (wilde Pointer, einige Anwendungen in Kernel, Abstimmung zwischen Programmierern etc.)

\subsubsection{Wechsel vom User Mode in Kernel Mode}
\textit{syscall Instruktion notwendig $\rightarrow$ Prozessor schaltet in Kernel-Mode um $\rightarrow$ setzt Instruction Pointer auf System Call Handler\\}
jede OS-Kernel-Funktion hat somit einen Code. Dieser wird in Register übergeben. Je nach Funktion in anderen Register weitere Infos.\\
Linux-Kernel nicht binärkompatibel wegen unterschiedlichen Calling Conventions (anderer syscall Code/anderes Register) $\rightarrow$ Appl. für jeden Kernel einzeln kompilieren, C-API (auf Quellcode, ABI = interface auf binary) verwenden

\subsection{Programmargumente}
Argumente vom OS in Speicherbereich des Programms als Array mit Pointern auf null-terminierte Strings\\
\prgc{main (int argc, char** argv)}: \prgc{argc} Anz. Argumente, \prgc{argv} Pointer auf Array mit Strings(\prgc{char*}), \prgc{argv[0]} Programmname!

\subsection{Umgebungsvariablen}
Umgebungsvar. vom OS in Speicherbereich des Programms kopiert als \textcolor{green}{Array mit Pointern} auf \textcolor{brown}{null-terminierte Strings} (wie Programmarg.)\\
Gemäss POSIX jeder Prozess eigene Umgebungsvariablen\\
environ $\rightarrow$\textcolor{green}{environ[0]}$\rightarrow$ \textcolor{brown}{Key0=Value0}\\
\textcolor{brown}{String:} \textcolor{blue}{PATH}=\textcolor{red}{/home/hsr/bin}, \textcolor{blue}{Key} (unique) \textcolor{red}{Value}\\
Umgebungsvariablen initial vom erzeugenden Prozess festgelegt (z.B. shell)

\subsubsection{API}
nie direkt über environ!\\
\prgc{char * getenv (const char * key)} Adresse 1. Zeichens, 0 nichts gefunden\\
\prgc{int setenv(const char *key, const char *value, int overwrite);}\\
overwrite != 0 ganzer Wert wird mit neuem String übschrieben\\
\prgc{int unsetenv(const char *key);} entfernt Umgebungsvariable\\
\prgc{int putenv (char * kvp)} ersetzt mit Pointer, keine Kopie (wie set)!\\

Einfügen von neuen Umgebungsvar.$\rightarrow$OS legt neuen grösseren Speicher an, kopiert Array dorthin