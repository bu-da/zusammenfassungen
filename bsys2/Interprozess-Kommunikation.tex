\section{Interprozess-Kommunikation (IPC)}
\subsection{Signale}
ermöglichen Unterbruch eines Prozesses von aussen\\
wird vom OS wie ein Interrupt behandelt
%\begin{itemize}
%    \item Unterbruch des laufenden Prozesses/(Threads)
%    \item Auswahl Signal-Handler Funktion
%    \item Ausführen Signal-Handler Funktion
%    \item Fortsetzen des Prozesses (falls Prozess nicht beendet)
%\end{itemize}

\subsubsection{Quelle von Signalen}
\textbf{Hardware/OS: }Ungültige Instruktion, Zugriff auf ungültigen Speicherbereich (\textit{segmentation fault}), Division durch 0
\textbf{Andere Prozesse: }Ctrl-C, \prgc{kill}-Kommando

\subsubsection{Signale behandeln}
Jeder Prozess pro Signal 1 Handler (bei Prozessbeginn Default-Handler)
\begin{tabular}{ll}
     Ignore-Handler & ignoriert Signal \\
     Terminate-Handler &  beendet Programm\\
     Abnormal-Terminate-Handler & beendet + Core Dump(Speicherauszug)
\end{tabular}
\textbf{Ausser SIGKILL \& SIGSTOP alle Handler überschreibbar}

\subsubsection{Wichtige Signale}
\textbf{Programmfehler}$\rightarrow$Abnormal-Terminate-Handler\\
SIGFPE Fehler in arithmetischer Operation SIGILL Ungültige Instruktion SIGSEGV Ungültiger Speicherzugriff SIGSYS Ungültiger Systemaufruf

\textbf{Prozesse abbrechen}$\rightarrow$Terminate-Handler\\
SIGTERM normale Beendigungsanfrage \bash{kill 1234}
SIGINT nachdrücklichere Aufforderung \bash{Ctrl-C}
SIGQUIT anormale Terminierung \bash{Ctrl-\ (Ctrl-Alt Gr-<)}
SIGABRT anormale Terminierung (vom Prozess selber bei Programmierfehler)
SIGKILL letzte Möglichkeit, kann nicht blockiert/ignoriert/abgefangen werden

\textbf{Stop und Continue}\\
SIGTSTP versetzt in Zustand \textit{stopped}, ähnlich zu \textit{waiting} \prgc{Ctrl-Z} 
SIGSTOP wie SIGTSTP, kann nicht abefangen/ignoriert werden
SIGCONT setzt Prozess fort

%\bash{kill 1234 5678} SIGTERM an Prozesse 1234 5678\\
%\bash{kill -Kill 1234} SIGKILL an Prozess 1234\\
%\bash{kill -l} listet alle möglichen Signale auf 

\subsubsection{Signalhandler ändern}
\begin{minted}{C}
int sigaction (//um Handler für Signal anzumelden
int signal, //Nr. des Signals, welches man handeln will
struct sigaction * new, 
struct sigaction * old)//0 wenn unerwünscht
\end{minted}
\begin{minted}{C}
struct sigaction{
void (* sa_handler )( int );//Handlerfunktion
sigset_t sa_mask;//alle zu blockierende Signale
int sa_flags;};
//sigset = Menge aller zu blockierender Signale
\end{minted}
%\begin{minted}{C}
%sigemptyset(sigset_t *set);
%sigfillset(sigset_t *set);
%sigaddset(sigset_t *set, int signal);//hinzufügen
%sigdelset(sigset_t *set, int signal);//entfernen
%sigismember(const sigset_t *set, int signal);//1 -> true
%\end{minted}

\subsection{Message-Passing}
%Implementationsabhängig: fixe/variable Nachrichtengrösse + folgende Punkte: %, Direkte/indirekte Kommunikation, Synchrone/asynchrone Kommunikation, Pufferung, Mit/ohne Prioritäten

\subsubsection{Direkte Kommunikation}
Sender muss Empfänger kennen \prgc{send (receiver, message)}\\
symmetrisches Empfangen: Empfänger muss Sender kennen %\prgc{receive (sender, message)}
Asymmetrisches Empfangen: Empfänger erhält ID in Out-Parameter (kennt Sender nicht) %\prgc{receive(id, message)}

\subsubsection{Indirekte Kommunikation}
Beide Teilnehmer müssen gleiche Mailbox/Port/Queue kennen\\
Mehrere Mailboxen zwischen Sender/Receiver möglich\\
%\prgc{send(Queue, message)} \prgc{receive(Queue, message)}\\
Queue gehört zu Prozess oder zu OS(Lösch/Erzeugmechanism.)

\subsubsection{Synchronisation}
blockierend (synchron)/nicht-blockierend(asynchron)\\
\textbf{synchrones Senden} Sender blockiert bis Nachricht empfangen\\
%\textbf{asynchrones Senden} Sender sendet und fährt weiter
\textbf{synchrones Empfangen} Empfänger blockiert bis Nachricht verfügbar\\
%\textbf{asynchrones Empfangen} 
$\rightarrow$Alle Kombinationen möglich (z.B. synchroner Sender/asynchroner Receiver)

\subsubsection{Rendezvous}
Sender \& Empfänger blockierend\\
OS kann direkt vom Sende- in Empfängerprozess kopieren (meistens ungepuffert)
(implizite Synch, Impl. Producer/Consumer-Problem)

%\subsubsection{Pufferung}
%\textit{Keine: }Queue-Länge = 0 | Sender muss blockieren
%\textit{Beschränkt: }$n$ Nachrichten können zwischengespeichert werden | Sender blockiert, wenn Queue voll
%\textit{Unbeschränkt}

%\subsubsection{Prioriäten}
%Empfänger holt Nachricht mit höchster Prio zuerst

\subsection{POSIX API}
\begin{itemize}
    \item Message-Queues vom OS
    \item variable Nachrichtenlänge, Maximum pro Queue einstellbar
    \item synchrone/asynchrone Verwendung
    \item Prioritäten
\end{itemize}

\begin{minted}{C}
mqd_t //Message-Queue-Descriptor
mqd_t mq_open (const char * name , int flags,
mode_t mode , struct mq_attr * attr);//attr=0 -> Default-Attr.
//Flags = 0_RDONLY, 0_CREAT, 0_NONBLOCK
int mq_close(mqd_t queue);//bleibt im OS bis Entfernung
int mq_unlink(const char * name);//wird entfernt, wenn keine Proz.
int mq_send ( mqd_t queue , const char * msg ,
size_t length , unsigned int priority);//blockiert wenn Queue voll
int mq_receive ( mqd_t queue , const char * msg,
size_t length , unsigned int * priority);//blockiert, wenn leer
//length mind. solange wie max. Grösse Nachricht
\end{minted}


