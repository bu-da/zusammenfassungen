\section{Meltdown}

\subsection{Ausgangslage}
\begin{itemize}
    \item Jeder Prozess hat eigenen virtuellen Adressraum
    \item Fordert ein Prozess einen OS-Service an, müsste man eigentlich den Kontext wechseln
    \item Macht er nicht, bleibt in aktuellem Kontext des Prozesses | Performancegründe!
    \begin{itemize}
        \item OS mappt alle Kernel-Daten in Adressraum des Prozesses
        \item Table-Config: nur OS kann daraufzugreifen
    \end{itemize}
    \item OS muss auf alle Prozesse zugreifen können
    \begin{itemize}
        \item Der OS-Kernel mappt den \underline{gesamten physischen} Hauptspeicher in jeden virtuellen Adressraum
    \end{itemize}
\end{itemize}
$\rightarrow$Ganzer Hauptspeicher kann ausgelesen werden

\subsection{Out-Of-Order Execution}
Moderne CPUs optimieren $\rightarrow$ Reihenfolge der Ausführung kann sich ändern
O3E möglich, auch wenn Befehl später nicht ausgeführt wird:\\
mov rax, [0x1234]\\
jz label\\
mov rbx, 9 \textit{Ergebnis wird verworfen, wenn übersprungen}\\
label:

\subsection{Seiteneffekte O3E}
\begin{minted}{C}
char d = 65;
int * p = 0;
*p = 1; //Exceptiom kann nicht an Adresse 0 schreiben
char c = array [d]; //wird spekulativ ausgeführt
\end{minted}
Cache kann man nicht auslesen, aber Zugriffszeitmessung möglich, kurz$\rightarrow$Zeile war im Cache

\subsection{Eigentlicher Angriff}
Array als Hilfsmittel\\
Array in Assembler: [Startadresse + index * Variablengrösse in Byte]
\begin{enumerate}
    \item clflush p für jedes Array-Element $\rightarrow$ Elemente werden aus Cache entfernt
    \item via O3E auf Array zugreifen, kurze Zugriffszeit $\rightarrow$Adresse mit verw. Daten gefunden
\end{enumerate}
\textbf{Array-Element-Grösse: }Array-Elemente auf verschiedene Pages verteilen \prgc{array[i * 4096]}, da Chache optimiert sein könnte (mehrere Zeilen miteinander lädt)

\subsection{Gegenmassnahmen}
\textbf{Kernel page-table isolation: }Verschiedene Page-Tables für Kernel/User-Mode $\rightarrow$ negative Ausw. auf Perfomance

\subsection{Spectre}
Ziel wie Meltdown. Weitere Eigenschaft wird ausgenutzt:
\subsubsection{Branch Prediction}
Prozis lernen ob Sprung erfolgt oder nicht. Muss Prozi auf Sprungbedingung warten$\rightarrow$Ausführung wahrscheinlicher Zweig

\subsubsection{Angriffsflächen}
\begin{itemize}
    \item Alle Prozesse, die auf gleichem Prozi, haben die gleichen Vorhersagen$\rightarrow$Vorhersagen können für andere Prozesse trainiert werden
    \item Opfer-Prozess muss zur Kooperation gezwungen werden $\rightarrow$Ziel im verworfenen Branch auf Speicher zugreifen
\end{itemize}







